
\documentclass{article}
\usepackage[landscape]{geometry}
\usepackage{url}
\usepackage{multicol}
\usepackage{amsmath}
\usepackage{esint}
\usepackage{amsfonts}
\usepackage{tikz}
\usetikzlibrary{decorations.pathmorphing}
\usepackage{amsmath,amssymb}

\usepackage{colortbl}
\usepackage{xcolor}
\usepackage{mathtools}
\usepackage{amsmath,amssymb}
\usepackage{enumitem}
\setlist{nosep}
\usepackage{hyperref}
\hypersetup{
    colorlinks=true,
    urlcolor=blue,
    }
\urlstyle{same}

%Russian-specific packages
%--------------------------------------
\usepackage[T2A]{fontenc}
\usepackage[utf8]{inputenc}
\usepackage[russian]{babel}
%--------------------------------------
\makeatletter

\newcommand*\bigcdot{\mathpalette\bigcdot@{.5}}
\newcommand*\bigcdot@[2]{\mathbin{\vcenter{\hbox{\scalebox{#2}{$\m@th#1\bullet$}}}}}
\makeatother

\advance\topmargin-.8in
\advance\textheight3in
\advance\textwidth3in
\advance\oddsidemargin-1.5in
\advance\evensidemargin-1.5in
\parindent0pt
\parskip2pt
\newcommand{\hr}{\centerline{\rule{3.5in}{1pt}}}
%\colorbox[HTML]{e4e4e4}{\makebox[\textwidth-2\fboxsep][l]{texto}

\begin{document}

\begin{center}{\huge{\textbf{Школа наставников 10 – Cheat Sheet v.1}}}\\
\href{https://github.com/egens/mentorschool\_cheatsheet}{https://github.com/egens/mentorschool\_cheatsheet}
\end{center}
\begin{multicols*}{3}

\tikzstyle{mybox} = [draw=black, fill=white, very thick,
    rectangle, rounded corners, inner sep=10pt, inner ysep=10pt]
\tikzstyle{fancytitle} =[fill=black, text=white, font=\bfseries]

%------------ Наставник в Практикуме ---------------
\begin{tikzpicture}
\node [mybox] (box){%
    \begin{minipage}{0.3\textwidth}
        \textbf{Наставник} — более опытный коллега. Он эмоционально поддерживает студентов, помогает в организационных вопросах, передаёт опыт и навыки, вовлекает в сообщество и делится культурой взаимодействия внутри него.\\
        \textbf{Подходы}
        \begin{enumerate}
          \item \textbf{Наставничество} Передать навыки (софт-скиллы и хард-скиллы), культуру взаимодействия, включить в сообщество.
          \item \textbf{Преподавание, тренинг}	Передать знания или навыки.
          \item \textbf{Бизнес-консалтинг}	Предложить экспертное решение в ответ на конкретный запрос.
          \item \textbf{Коучинг}	Помочь клиенту раскрыть потенциал и достичь цели.
          \item \textbf{Психотерапия}	Помочь клиенту решить личностные проблемы.
        \end{enumerate}
        \textbf{Наставничество}
        \begin{enumerate}
          \item Оценить навыки и показатели сотрудника
          \item Закрыть пробелы в знаниях
          \item Поделиться инсайдерской информацией об особенностях работы в этой отрасли
          \item Помочь составить план дальнейших действий
          \item Дать обратную связь по прогрессу
        \end{enumerate}
        \textbf{Функции в практикуме}
        \begin{enumerate}
          \item Образовательная функция
          \item Функция передачи опыта
          \item Функция мотивации и поддержки
        \end{enumerate}
        Помогают – куратор, ревьюер, поддержка. Иногда преподаватели и старшие студенты.\\
        \textbf{Ключевые метрики практикума}
        \begin{enumerate}
          \item Показатель доходимости (Completion rate)
          \item Показ. удовлетворённости (Satisfaction rate)
          \item Показатель трудоустройств
          \item Показатель прохождения испытательного
        \end{enumerate}
        \textbf{Опережающие показатели}
        \begin{enumerate}
          \item Показатель доходимости
          \item Причины академов и возвратов
          \item Мотивация учиться дальше
          \item Оценка работы наставника
          \item Уверенность в освоенном материале
        \end{enumerate}
    \end{minipage}
};
%------------ Наставник в Практикуме Header ---------------------
\node[fancytitle, right=10pt] at (box.north west) {Наставник в Практикуме};
\end{tikzpicture}

%------------ Здоровая коммуникация ---------------
\begin{tikzpicture}
\node [mybox] (box){%
    \begin{minipage}{0.3\textwidth}
        \textbf{Cамонаправленное обучение} (Джеральд Гроу)\\
        Зависимый $\rightarrow$ Заинтересованный $\rightarrow$ \\
        Вовлечённый $\rightarrow$ Самонаправленный\\
        \textbf{Озвучивать и повторять договоренности.}\\
        \textbf{Токсичная коммуникация}
        \begin{enumerate}
          \item Оценка, обобщение – Ты разгильдяй
          \item Принижающие сообщения
          \item Навязывание решений – Делай, что говорю
          \item Манипуляция – подмена следствий, угроза, давление на эмоции
          \item Газлайтинг – Этого не было
        \end{enumerate}
        \textbf{Как правильно}
        \begin{enumerate}
          \item Вместо оценок и обобщений говорить о конкретных ситуациях и проявлениях.
          \item Принижающие высказывания вроде «Ты что, самый умный?» заменить на вопросы «Объясни, пожалуйста, логику своего решения?».
          \item Вместо навязывания решений предлагать студентам делать выбор и аргументировать.
        \end{enumerate}
        \textbf{Алгоритм ненасильственного общения}
        \begin{enumerate}
          \item Описать факт или наблюдение
          \item Предположить чувства или переживания
          \item Назвать универсальную потребность
          \item Сформулировать возможное решение
        \end{enumerate}
        \textbf{Потребности в ...}\\
        Сотрудничестве – Ясности – Уважении – Безопасности – Поддержке – Понимании\\
        \textbf{Губительное сочувствие} – из соображений заботы не дает объективную обратную связь.\\
        \textbf{Манипулятивная неискренность} – не высказывает критику из личных мотивов. Например не хочет объяснять всё заново.\\
        \textbf{Вызывающая агрессия} – открыто реагирует и при этом преследует личную выгоду. Например критикует, чтобы показать своё превосходство.\\
        \textit{\textbf{Радикальная откровенность} – честно говорит студентам об их ошибках. Заботится, чтобы не потеряли мотивацию. Формулирует ОС так, что  хочется учиться дальше.}\\
        \textbf{Правила ОС.} Конкретная и развивающая. Наставник не закрывает глаза на ошибки и рассказывает, как в следующий раз их можно избежать.\\
        Замечания высказывают лично, похвалу — публично.
    \end{minipage}
};
%------------ Здоровая коммуникация Header ---------------------
\node[fancytitle, right=10pt] at (box.north west) {Здоровая коммуникация};
\end{tikzpicture}

%------------ Мотивация по Райану — Деси ---------------
\begin{tikzpicture}
\node [mybox] (box){%
    \begin{minipage}{0.3\textwidth}
        \textbf{Три основные потребности}
        \begin{enumerate}
          \item В автономии
          \item В компетентности и эффективности
          \item В значимых межличностных отношениях, в привязанности
        \end{enumerate}
        \textbf{Мотивация}
        \begin{enumerate}
          \item Внешняя – деятельность для удовлетворения запросов извне, получения вознаграждения.
          \item Автономная – похожа на внешнюю, но с более высоким уровнем автономии.
          \item Внутренняя – ее источник – сам человек, его интересы и стремления.
        \end{enumerate}
        Для того, чтобы внешняя мотивация превратилась в автономную, студенты должны принять правила и положения, понять их роль в достижении цели. 
        \textbf{Основа мотивирующего ответа}\\
        Опора на внутреннюю или автономную мотивацию.
        \begin{enumerate}
          \item Личная большая цель
          \item Чувство успеха
          \item Потребность в познании
        \end{enumerate}
        \textbf{Внешняя мотивация}
        \begin{enumerate}
          \item "От" – на отрицательных стимулах
          \item "К" – на положительных
          \item Подкрепление, чтобы поощрить желательное поведение
        \end{enumerate}
        Чтобы соблюдали дисциплину, выполняли краткосрочные задачи, действовали по шаблону.\\
        Не рекомендуется, если цель — помочь стать самостоятельными и научить креативно решать задачи.
        \textbf{Эффект Даннинга — Крюгера}
        \begin{enumerate}
          \item Медовый месяц
          \item Скала смятения
          \item Пустыня отчаяния
          \item Подъём благоговения
        \end{enumerate}
        \textbf{Возвращение к реальности}
        \begin{enumerate}
          \item Фактическое перечисление ошибок и побед
          \item Определение того, что ошибки — норма
          \item Концентрация на текущей задаче
        \end{enumerate}
        
    \end{minipage}
};
%------------ Мотивация по Райану — Деси Header ---------------------
\node[fancytitle, right=10pt] at (box.north west) {Мотивация по Райану — Деси};
\end{tikzpicture}

%------------ Best practice образовательного процесса ---------------
\begin{tikzpicture}
\node [mybox] (box){%
    \begin{minipage}{0.3\textwidth}
        \textbf{Реакции на вопросы}
        \begin{enumerate}
          \item Экспертный ответ – на сложные вопросы
          \item Менторский вопрос – натолкнуть на решение
          \item Переадресация – если вне компетенции
        \end{enumerate}
        \textbf{Картирование} — соотнесение с учебными или рабочими задачами, с динамикой обучения.\\
        \textbf{Декомпозиция} — это расщепление большой задачи на более мелкие.\\
        \textbf{Картирование поможет}
        \begin{enumerate}
          \item Объяснить место темы в программе обучения
          \item Соотнести тему с задачами в профессии
          \item Поддержать студента в сложном пути
          \item Побороть предубеждения из прошлого опыта
        \end{enumerate}
        \textbf{Виды объяснений}
        \begin{enumerate}
          \item Сравнение
          \item Аналогия
          \item Метафора
          \item Сторителлинг
          \item Визуализация
        \end{enumerate}
        \textbf{Традиционная модель обучения} - от теории к практике
        \begin{enumerate}
          \item Презентация
          \item Проверка знания
          \item Домашнее задание
          \item Оценка
        \end{enumerate} 
        \textbf{Цикл Колба} - от опыта к теории через рефлексию 
        \begin{enumerate}
          \item Получение непосредственного опыта
          \item Анализ полученного опыта
          \item Теоретическое обобщение
          \item Проверка теории на практике
        \end{enumerate} 
    \end{minipage}
};
%------------ Best practice образовательного процесса Header ---------------------
\node[fancytitle, right=10pt] at (box.north west) {Best practice образовательного процесса};
\end{tikzpicture}

%------------ Мастерство ведения онлайн-воркшопа ---------------
\begin{tikzpicture}
\node [mybox] (box){%
    \begin{minipage}{0.3\textwidth}
        \textbf{Цель и образовательные результаты} 
        \begin{enumerate}
          \item Какую проблему студентов решает?
          \item Как именно он поможет твоим студентам?
        \end{enumerate}
        \textbf{Типы воркшопов} 
        \begin{enumerate}
          \item Лайвкодинги или демо — связать теорию из тренажёра с практической работой.
          \item Q\&A — ответить на накопившиеся вопросы.
          \item AMA — неформальный, установить эмоциональный контакт с группой, наладить общение и преодолеть кризисы.
        \end{enumerate}
        \textbf{Введение (10–15\% времени)}
        \begin{enumerate}
          \item Самопрезентация
          \item Тема, цель и план воркшопа
          \item Актуализация темы – проблемматика, картирование, кейс
          \item Сбор ожиданий
        \end{enumerate} 
        \textbf{Основная часть (до 75\% времени)}
        \begin{enumerate}
          \item Цикл Колба
          \item Связки-переходы
          \item Презентация
          \item Вопросы
          \item Коллаборация онлайн
        \end{enumerate} 
        \textbf{Заключение (10–15\% времени)}
        \begin{enumerate}
          \item Поинтересоваться состоянием участников
          \item Обратиться к целям и ожиданиям студентов
          \item Собрать личные итоги
          \item Пройтись по плану занятия
          \item Проверить полученные знания
          \item Сказать напутственные слова
          \item Дать домашнее задание
        \end{enumerate} 
        \textbf{Правила}
        \begin{enumerate}
          \item Начинать вовремя — и после перерыва тоже
          \item Придерживаться правила одного микрофона
          \item Иван Дорн — «не надо стесняться»
          \item Участвовать в воркшопе с камерой
          \item Уважительно относиться друг к другу
        \end{enumerate} 
        \textbf{Анонс}
        \begin{enumerate}
          \item «Почему» — какую главную ценность для себя вынесут участники. Ради чего  прийти?
          \item «Как» — формат. Программировать онлайн? Решать кейсы? Соревноваться группами?
          \item «Что» — какие темы и инструменты ты планируешь разобрать.
        \end{enumerate} 
    \end{minipage}
};
%------------ Мастерство ведения онлайн-воркшопа Header ---------------------
\node[fancytitle, right=10pt] at (box.north west) {Мастерство ведения онлайн-воркшопа};
\end{tikzpicture}

%------------ Принципы онлайн-взаимодействия ---------------
\begin{tikzpicture}
\node [mybox] (box){%
    \begin{minipage}{0.3\textwidth}
        \textbf{Рецепт} Интерактив от ведущего → Реакция аудитории → Обратная связь от ведущего на реакцию\\
        \textbf{Реакция аудитории}
        \begin{enumerate}
          \item Писать в чат
          \item Отвечать с использованием онлайн-инстр.
          \item Высказываться в небольшой группе
          \item Молчать
          \item Уходить из трансляции
        \end{enumerate}
        \textbf{Обратная связь на реакцию аудитории}
        \begin{enumerate}
          \item Поблагодари участников, признай достижения
          \item Сделай комплимент, где это уместно
          \item Зачитай комментарии или их часть
          \item Поддержи чьё-то мнение
          \item Обратись к конкретному участнику
          \item Запомни и далее вернись к чьему-то опыту
          \item Отвечай на вопросы
        \end{enumerate}
        \textbf{Универсальные принципы онлайн взаимод.}
        \begin{enumerate}
          \item Регулярно запускай цикл взаимод. (5-10 мин)
          \item Чередуй форматы (+, вопрос, оценка)
          \item Проверь инструкцию
          \item Следи за исполнением твоей просьбы
          \item Вопросы, на которые не страшно отвечать
        \end{enumerate}
        \textbf{Инструкция}
        \begin{enumerate}
          \item Смысл упражнения
          \item Конкретная цель или задача
          \item Правила или ограничения
          \item Образ результата
          \item Возможность задать вопросы
        \end{enumerate}
        \textbf{Вопросы, на которые у тебя нет ответа}
        \begin{enumerate}
          \item Парковать
          \item Переадресовывать
          \item Брать паузу
          \item Предложить обсуждение в чате
          \item Вопрос в треде к следующей консультации
          \item Дать отсылку на внешний источник
        \end{enumerate}
    \end{minipage}
};

%------------ Принципы онлайн-взаимодействия Header ---------------------
\node[fancytitle, right=10pt] at (box.north west) {Принципы онлайн-взаимодействия};
\end{tikzpicture}

%------------ Шпаргалка по интерактивам ---------------
\begin{tikzpicture}
\node [mybox] (box){%
    \begin{minipage}{0.3\textwidth}
        \textbf{Предложи представить новую ситуацию:}
        \begin{itemize}
          \item «Представьте себе, что… Как бы вы себя чувствовали?».
        \end{itemize}
        \textbf{Обратись к опыту участников:}
        \begin{itemize}
          \item «Поделитесь опытом, если вы…».
          \item «Что вы можете посоветовать?».
        \end{itemize}
        \textbf{Проверь полученные знания:}
        \begin{itemize}
          \item «Давайте посмотрим, что вы усвоили…».
          \item «Кто скажет, как, сколько, зачем…».
        \end{itemize}
        \textbf{Дай задание:}
        \begin{itemize}
          \item С использованием слайда. Например, попроси решить кейс, найти и исправить ошибку.
          \item Устно. Озвучь задачу и попроси участников написать ответы.
        \end{itemize}
        \textbf{Попроси проголосовать за варианты:}
        \begin{itemize}
          \item «Выберите и напишите номер варианта...».
        \end{itemize}
        \textbf{Убедись, что участники следуют за мыслью:}
        \begin{itemize}
          \item «Если вам понятно, поставьте „+“».
        \end{itemize}
        \textbf{Добавь геймификацию:}
        \begin{itemize}
          \item Ставки («Правда/ложь», «Угадайте, что»).
          \item Гадание («Выберите один из вариантов и объясните, почему…»).
          \item Соревнования («Кто быстрее или необычнее выполнит задание», «Кто первым угадает»).
          \item Баттл между командами, если есть возможность для командной работы.
          \item Помощь («Кто может подсказать/помочь»).
          \item Задание с подсказками, когда даём наводки по мере прохождения.
          \item Открытие контента по мере проведения ВШ.
          \item Квест. Участники соревнуются на время или проходят задания одной командой.
          \item Баллы за действия, приз в конце. Статусы.
        \end{itemize}
        \textbf{Встрой активные форматы взаимодействия:}
        \begin{itemize}
          \item Вывод участников в эфир с презентацией.
          \item Совместная работа, напр. написание кода.
          \item Мозговой штурм на время.
        \end{itemize}
    \end{minipage}
};
%------------ Шпаргалка по интерактивам Header ---------------------
\node[fancytitle, right=10pt] at (box.north west) {Шпаргалка по интерактивам};
\end{tikzpicture}

%------------ Чек-лист воркшопа ---------------
\begin{tikzpicture}
\node [mybox] (box){%
    \begin{minipage}{0.3\textwidth}
        \textbf{За 2‒3 дня до воркшопа необходимо:}
        \begin{itemize}
          \item Собрать ожидания о предстоящем воркшопе: о чём студентам важно поговорить, про что они хотели бы послушать.
          \item Доделать план воркшопа, продумать тайминг и подготовить презентацию. Не работать над презентацией в ночь перед встречей.
        \end{itemize}
        \textbf{В день воркшопа:}
        \begin{itemize}
          \item Проверь, что у тебя работает интернет, запусти «Зум», включи камеру.
          \item Лучше, если у тебя будут заряженные ноутбук и телефон. Свет всегда отключают в самое неподходящее время.
          \item Обязательно открой каждый пример, который собираешься показывать. Убедись, что всё отображается корректно.
          \item Проверь фон и свет. Если фон смущает, можно пользоваться фоном Практикума.
          \item Запиши план воркшопа с таймингом, держи его перед глазами.
          \item Подготовь таймер, чтобы засекать время. 
        \end{itemize}
    \end{minipage}
};
%------------ Чеклист воркшопа Header ---------------------
\node[fancytitle, right=10pt] at (box.north west) {Чеклист воркшопа};
\end{tikzpicture}

%------------ Конфликт и сопротивление ---------------
\begin{tikzpicture}
\node [mybox] (box){%
    \begin{minipage}{0.3\textwidth}
        \textbf{Конфликт} — противостояние, вызванное тем, что у каждой стороны свой интерес.\\
        \textbf{Сопротивление} — когда одна сторона не хочет вовлекаться в то, что предлагает другая сторона.
        \begin{enumerate}
          \item Картируй разбираемые темы
          \item Давай чёткие и ясные инструкции
          \item Объясняй сложный материал по разному
          \item Давай аргументированную обратную связь
          \item Хвали за дела, не касаясь личности
        \end{enumerate}
        \textbf{Обработка сопротивления}
        \begin{enumerate}
          \item Какая из базовых потребностей не закрыта
          \item Определи со студентом шаги для ее закрытия
          \item Поблагодари за открытость и сотрудничество
          \item Сделай шаги или обозначь сроки
        \end{enumerate}
        \textbf{Ошибки при обработке сопротиления}
        \begin{enumerate}
          \item Выход из роли наставника в психологические
          \item Сильные негативные эмоции в ответ
          \item Не менять поведение при росте сопр.
        \end{enumerate}
        \textbf{Типы конфликтов и коммуникация}
        \begin{center}
        \begin{tabular}{ |c|c|c| } 
         \hline
          & \textbf{Конструкт.} & \textbf{Неконструкт.} \\ 
         \textbf{Реалист.} & Адекватная & Осложнённая \\ 
         \textbf{Нереалист.} & Осложнённая & Токсичная \\ 
         \hline
        \end{tabular}
        \end{center}
    \end{minipage}
};
%------------ Конфликт и сопротивление Header ---------------------
\node[fancytitle, right=10pt] at (box.north west) {Конфликт и сопротивление};
\end{tikzpicture}

%------------ Как действовать при конфликте ---------------
\begin{tikzpicture}
\node [mybox] (box){%
    \begin{minipage}{0.3\textwidth}
        \textbf{Обработка конфликта}
        \begin{enumerate}
          \item \textbf{Настрой на сотрудничество}
          \item Обозначь суть конфликта и общие цели
          \item Обсуди точки зрения сторон
          \item Сформулируй совместное решение
        \end{enumerate}
        \textbf{Отделение личности от роли наставника}
        \begin{enumerate}
          \item Осознай роли обеих сторон
          \item Управляй конфликтом исходя из своей роли
          \item Помни общую цель, ищи способы достижения
          \item Посоветуй к кому обратиться, если решение конфликта не в твоей зоне ответственности
        \end{enumerate}
    \end{minipage}
};
%------------ Как действовать при конфликте Header ---------------------
\node[fancytitle, right=10pt] at (box.north west) {Как действовать при конфликте};
\end{tikzpicture}

%------------ Сложные ситуации ---------------
\begin{tikzpicture}
\node [mybox] (box){%
    \begin{minipage}{0.3\textwidth}
        \textbf{Ошибки}
        \begin{enumerate}
          \item Пассивно-агрессивный тон и обесценивание
          \item Не признавать вину и переход на личности
          \item Предлагать студенту «успокоиться»
          \item Обобщение и уход в позицию виноватого
        \end{enumerate}
        \textbf{Best practrice}
        \begin{enumerate}
          \item Выслушать и сохранять спокойствие
          \item Признать значимость проблемы
          \item Спросить у студента, какое решение он видит
          \item Подчеркнуть общность с собеседником
        \end{enumerate}
        \textbf{Ловушки}
        \begin{enumerate}
          \item Потерять смысл что-то делать
          \item Попасть в зависимые отношения – контролер, спасатель, жертва
          \item Ассоциировать себя с чужими проблемами
          \item Стремиться сделать всё слишком идеально
        \end{enumerate}
        \textbf{Признаки токсичной коммуникации}
        \begin{enumerate}
          \item Преимущ. публичное высказывание негатива.
          \item Неконстр. критика, переход на личности.
          \item Угрозы придать проблему огласке или шантаж обращением в инстанции.
          \item Постоянный поиск новых конфликтов вне зависимости от действий команды.
        \end{enumerate}
        С токсичными студентами работает куратор. Если ты подозреваешь, что в общении могут быть проявления токсичной коммуникации:
        \begin{enumerate}
          \item Вежливо заверши разговор.
          \item Сделай скриншот переписки или найди нужный момент в записи воркшопа.
          \item Обратись к куратору и лиду сопровождения.
        \end{enumerate}
        Позиции вины и силы vs \textbf{Партнерская позиция}.
    \end{minipage}
};
%------------ Сложные ситуации Header ---------------------
\node[fancytitle, right=10pt] at (box.north west) {Сложные ситуации};
\end{tikzpicture}

%------------ Методичка по сложным ситуациям ---------------
\begin{tikzpicture}
\node [mybox] (box){%
    \begin{minipage}{0.3\textwidth}
        \textbf{Повышенная активность отдельных участников}
        \begin{enumerate}
          \item Поблагодари за участие, дай слово другим.
          \item Обратись лично к тем, кто мало участвовал, и попроси их высказаться.
        \end{enumerate}
        \textbf{Споры с ведущим}
        \begin{enumerate}
          \item Выслушай и дай ответ с пониманием точки зрения каждого.
          \item Нейтралитет к другим участникам.
          \item Напомни цель, которую поставили с группой.
          \item Подведи итог, резюмируй.
        \end{enumerate}
        \textbf{Прямой негатив от участников}
        \begin{enumerate}
          \item Сохраняй спокойствие.
          \item Задай вопрос, чтобы понять, в чём проблема.
          \item Не переходи на личности, решай ситуацию, опираясь на ваши роли и договорённости.
        \end{enumerate}
        \textbf{Опоздания}
        \begin{enumerate}
          \item Начинай вовремя.
          \item Поблагодари тех, кто пришёл, но не ставь опоздавших в неловкое положение.
          \item Договаривайся не только о промежутке (через 10м), но и точном времени нач. (в 20:15).
        \end{enumerate}
        \textbf{Пассивность участников}
        \begin{enumerate}
          \item Начни с закрытых вопросов «на поднятие руки»: «Кто хочет сменить профессию? Поднимите руку», «Кто согласен, что это важная тема? Поставьте „плюс“ в чате».
          \item Задавай вопросы, поощряй участие.
          \item Обращайся по именам.
          \item Руководи очерёдностью высказываний.
        \end{enumerate}
    \end{minipage}
};
%------------ Методичка по сложным ситуациям Header ---------------------
\node[fancytitle, right=10pt] at (box.north west) {Методичка по сложным ситуациям};
\end{tikzpicture}

\columnbreak

%------------ Комьюнити в Практикуме ---------------
\begin{tikzpicture}
\node [mybox] (box){%
    \begin{minipage}{0.3\textwidth}
        \begin{enumerate}
          \item Сообщество помогает достигнуть целей
          \item У участн. складывается общая идентичность
          \item Практики, принятые в сообществе, воспроизводятся без внешнего продюсирования
        \end{enumerate}
        \textbf{Советы по формированию комьюнити}
        \begin{enumerate}
          \item Делись примерами профессиональных сообществ и преимуществ принадлежности к ним.
          \item Расскажи, как взаимопомощь поможет освоить материал и достигнуть целей. Лучший способ — обсуждать материалы и делиться логикой размышлений. А не списывать.
          \item Приветствуй дискуссии и вопросы в чате.
          \item Делись дополнительными материалами, кейсами из профессиональной жизни, примерами выстраивания карьеры. По опыту Практикума, студенты в разы активнее осваивают новую профессию, если их наставник становится не только формальным, но и неформальным лидером группы.
        \end{enumerate}
    \end{minipage}
};
%------------ Комьюнити в Практикуме Header ---------------------
\node[fancytitle, right=10pt] at (box.north west) {Комьюнити в Практикуме};
\end{tikzpicture}

\columnbreak

%------------ Общение в чате ---------------
\begin{tikzpicture}
\node [mybox] (box){%
    \begin{minipage}{0.3\textwidth}
        \textbf{Фоллоу-ап по итогам воркшопа}\\
        Приветствие → Содержание встречи в двух-трёх предложениях → Подробное описание договорённостей\\
        \textbf{Ответы на вопросы студентов в общем чате}
        \textbf{Личные посты в общем чате}
        \textbf{Сложности письменной коммуникации}
        \begin{enumerate}
          \item Из-за того, что коммуникация отсрочена, у студентов появляется ощущение покинутости. Особенно если они приходят с «горящей» проблемой, испытывают сильные эмоции.
          \item Сложно интерпретировать реакцию, эмоции, когда не видишь человека.
          \item Можно ошибиться словом — и исказить смысл. Это приведёт к конфликту.
          \item Не всегда ясно, понял ли тебя собеседник.
        \end{enumerate}
        \textbf{Правила}
        \begin{enumerate}
          \item Общайся доброжелательно
          \item Реагируй на сообщ., регламентируй паузы
          \item Разложи по полочкам, форматируй сообщ.
          \item Придерживайся структуры – приветствие, суть, вопрос или призыв к действию
          \item Переспрашивай
          \item Пиши проще
          \item Избегай двусмысленных шуток
          \item Прочитай перед тем, как отправить
        \end{enumerate}
        
    \end{minipage}
};
%------------ Общение в чате Header ---------------------
\node[fancytitle, right=10pt] at (box.north west) {Общение в чате};
\end{tikzpicture}

\end{multicols*}
\end{document}
